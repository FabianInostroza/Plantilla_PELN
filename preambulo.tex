% !TEX TS-program = pdflatex
% !TEX encoding = UTF-8 Unicode
% !TEX spellcheck = es_CL
% !TEX root = Plantilla_PELN.tex
% Creado por Fabian Inostroza, 2015
\usepackage{mathtools}
\usepackage{amsfonts}
\usepackage{amssymb}
\usepackage{fancyhdr}

\usepackage{xcolor}

\usepackage{graphicx}
\usepackage[hypcap=true]{caption}%
%\captionsetup{labelfont=bf} % solo etiqueta en negrita
\captionsetup{font={bf,small}}% todo el titulo en negrita
\usepackage[hypcap=true]{subcaption} % reemplaza a subfig y 

%\usepackage[titletoc,toc,title]{appendix}
%\usepackage[titletoc,toc]{appendix}

\usepackage{longtable}

\definecolor{tableau10_1}{RGB}{31,119,180}
\definecolor{tableau10_2}{RGB}{255,127,14}
\definecolor{tableau10_3}{RGB}{44,160,44}
\definecolor{tableau10_4}{RGB}{214,39,40}
\definecolor{tableau10_5}{RGB}{148,103,189}
\definecolor{tableau10_6}{RGB}{140,86,75}
\definecolor{tableau10_7}{RGB}{227,119,194}
\definecolor{tableau10_8}{RGB}{127,127,127}
\definecolor{tableau10_9}{RGB}{188,189,34}
\definecolor{tableau10_10}{RGB}{23,190,207}
\definecolor{gray75}{gray}{0.75}

\definecolor{MSBlue}{rgb}{.204,.353,.541}
\definecolor{MSLightBlue}{rgb}{.31,.506,.741}

% para mostrar algoritmos
\usepackage[ruled,noend]{algorithm2e}
\SetKwIF{If}{ElseIf}{Else}{si}{}{sino si}{sino}{fin si}
%\SetKw{Return}{devolver}
\SetKwFor{For}{para}{hacer}{fin para}
%\SetKwFor{ForEach}{para cada}{hacer}{fin para}
%\SetKwFor{ForAll}{repetir}{veces}{fin repetir}
\SetKwFor{While}{mientras}{hacer}{fin mientras}

\usepackage{booktabs}
%\usepackage{sectsty}
%\chapterfont{\Large}
%\chapterfont{\color{tableau10_1}}
%\sectionfont{\color{tableau10_1}}
%\allsectionsfont{\color{tableau10_1}} % equivalente a arriba

\newcommand{\hsp}{\hspace{20pt}}
\newcommand{\chaprule}{\titlerule[1.5pt]}
\usepackage{titlesec}
\titlespacing*{\chapter}{0pt}{0pt}{20pt}
\titleformat{\chapter}[hang] % Customise the \section command
{\fontsize{18}{21}\selectfont\sffamily\bfseries\raggedright}%
{\chaptertitlename~\thechapter.}{1em}% sections, like 'Section ...'
{} % Can be used to insert code before the heading
[\chaprule] % Inserts a horizontal line after the heading 

\titleformat{\section}
{\fontsize{16}{19}\selectfont\bfseries\raggedright}
{\thesection}{1em}{}

\titleformat{\subsection}
{\fontsize{14}{17}\selectfont\bfseries\raggedright}
{\thesubsection}{1em}{}

\titleformat{name=\section,numberless}[hang]
  {\fontsize{16}{19}\selectfont\sffamily\bfseries\raggedright}
  {}
  {0em}
  {}[]

% Capitulos sin numero sin regla inferior
\titleformat{name=\chapter,numberless}[hang]
  {\fontsize{18}{21}\selectfont\sffamily\bfseries\raggedright}
  {}
  {0em}
  {}[]


\renewcommand{\headrulewidth}{0pt}%
\fancyhf{}
\fancyhead[r]{\thepage}


% activar solo en la version final
% para que la compilacion no tome tanto tiempo
%\usepackage[pages=some]{background}
\usepackage{background}

\backgroundsetup{
scale=1,
color=black,
opacity=0.05,
angle=0,
contents={%
  \includegraphics[scale=0.5]{logoudec}
  }%
}

\newcommand{\monthname}{%
  \ifcase\the\month
  \or Enero% 1
  \or Febrero% 2
  \or Marzo% 3
  \or Abril% 4
  \or Mayo% 5
  \or Junio% 6
  \or Julio% 7
  \or Agosto% 8
  \or Septiembre% 9
  \or Octubre% 10
  \or Noviembre% 11
  \or Diciembre% 12
  \fi}

%\usepackage{xwatermark}

%\DeclareTextFontCommand{\texthv}{\fontencoding{T1}\fontfamily{cmss}\fontseries{bx}\fontshape{n}\selectfont}
%\DeclareTextFontCommand{\textcmss}{\usefont{OT1}{cmss}{bx}{n}}
\newcommand{\HRule}{\rule{\linewidth}{0.5mm}}

%\usepackage{tocloft}
%\cftsetindents{section}{0em}{3em}
%\cftsetindents{subsection}{3em}{4em}

%%%%%%%%%%%%%%%%%%%%%%%%%%%%%%%%%%%%%%%%%%%%%%%%%%%%%%%%%%%
%% FUENTES
%%%%%%%%%%%%%%%%%%%%%%%%%%%%%%%%%%%%%%%%%%%%%%%%%%%%%%%%%%%
\usepackage[T1]{fontenc}
% si se usa codificacion T1 y se prefiere
% la fuente textsf en negrita y mas gruesa,
% cargar el paquete lmodern, de lo contrario
% se usa otra fuente
% http://tex.stackexchange.com/questions/227063/fontenc-changes-sans-serif-bold-font-in-koma-script
%\usepackage{lmodern}
% si se usa tgpagella esto es solo para cargar las 
% fuentes matematicas
%\usepackage[sc]{mathpazo}
%\linespread{1.05}

%\usepackage{tgpagella}
%\usepackage[T1]{fontenc}

%\usepackage[minionint,textlf]{MinionPro}
%\renewcommand{\sfdefault}{Myriad-LF}

\usepackage{newtxtext} % fuente times, texto
%\usepackage{newtxmath} % fuente times, math
%\usepackage[varg,cmintegrals, cmbraces]{newtxmath}
\usepackage[cmintegrals, cmbraces, uprightGreek]{newtxmath}

%\usepackage{helvet}
%\renewcommand{\familydefault}{\sfdefault}

\usepackage{microtype}

\usepackage{listings}
\lstset{breaklines=true,basicstyle=\small\ttfamily,} 

% Corrige la separacion entre el numero de la figura/tabla y
% el texto en el indice cuando los numeros son muy
% largos (ej: 10.10), http://tex.stackexchange.com/a/22984
\makeatletter
\renewcommand*\l@figure{\@dottedtocline{1}{1.5em}{2.8em}}% 3em instead of 2.3em
\let\l@table\l@figure
\makeatother

\makeatletter
\let\Title\@title
\let\Author\@author
\makeatother

\usepackage{setspace}
%\linespread{1.5}

\makeatletter
\newcommand{\ProfGuia}[1]{\renewcommand\@ProfGuia{#1}}
\newcommand\@ProfGuia{}
\makeatother

% numerar ecuaciones como seccion.numero
\numberwithin{equation}{chapter}
\numberwithin{figure}{chapter}
\numberwithin{table}{chapter}
%\renewcommand{\figurename}{Figura}
%\renewcommand{\tablename}{Tabla}

%\usepackage{wrapfig}

%%%%%%%%%%%%%%%
%% BIBLIOGRAFIA
%%%%%%%%%%%%%%%%%%%%%%%%%%%%%%%%%%%%
%\renewcommand{\refname}{Bibliografía}
%\renewcommand{\bibname}{Bibliografía}
%\usepackage[nottoc,notlot,notlof]{tocbibind}

%%%%%%%%%%%%%%%%%%%%%%%%%%%%%%%%%%%%%%%%%%%%%%%%%%%%%%%%%%%%
%%              COMANDOS PROPIOS
%%%%%%%%%%%%%%%%%%%%%%%%%%%%%%%%%%%%%%%%%%%%%%%%%%%%%%%%%%%%
\newcommand{\vect}[1]{\mathbf{#1}}
\newcommand{\vects}[1]{\boldsymbol{#1}}
\DeclareMathOperator{\adj}{adj}
\DeclareMathOperator{\diag}{diag}
\newcommand{\Adj}[1]{\adj\{#1\}}
\newcommand{\Diag}[1]{\diag\{#1\}}
\newcommand{\Real}[1]{\operatorname{Re}\{#1\}}
\newcommand{\Imag}[1]{\operatorname{Im}\{#1\}}
\newcommand{\rango}[1]{\operatorname{rango}\{#1\}}
%\newcommand{\det}[1]{\operatorname{det}\{#1\}}
%\newcommand{\arg}[1]{\operatorname{arg}\{#1\}}
\newcommand{\tr}[1]{\operatorname{tr}\{#1\}}

% Apéndice -> Anexo
\addto{\captionsspanish}{\renewcommand*{\appendixname}{Anexo}}

%\DeclareUnicodeCharacter{00B0}{\ensuremath{^{\circ}}}

\usepackage[pagebackref=true]{hyperref}
\hypersetup{
    bookmarks=false,         % show bookmarks bar?
    unicode=true,          % non-Latin characters in Acrobat’s bookmarks
%    plainpages=false,
%    pdfpagelabels=true,
%    pdftoolbar=true,        % show Acrobat’s toolbar?
%    pdfmenubar=true,        % show Acrobat’s menu?
    pdffitwindow=false,     % window fit to page when opened
    pdfstartview={FitH},    % fits the width of the page to the window
    pdftitle={Proyecto Electrónico: \Title},    % title
    pdfauthor={\Author},     % author
%    pdfsubject={Subject},   % subject of the document
%    pdfcreator={Creator},   % creator of the document
%    pdfproducer={Producer}, % producer of the document
%    pdfkeywords={keyword1} {key2} {key3}, % list of keywords
    pdfnewwindow=true,      % links in new window
    colorlinks=\colorlinks,       % false: boxed links; true: colored links
    linkcolor=tableau10_1,          % color of internal links (change box color with linkbordercolor)
    linkbordercolor=tableau10_10,
    citecolor=tableau10_3,        % color of links to bibliography
    filecolor=magenta,      % color of file links
    urlcolor=tableau10_1           % color of external links
}
